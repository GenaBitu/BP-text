\begin{onehalfspace}
	\noindent \textit{Název práce:}

	\noindent \textbf{Hierarchické modely síťového provozu}
\end{onehalfspace}

\bigskip

\noindent \textit{Autor:} Marek Dědič

\bigskip

\noindent \textit{Obor:} Matematická informatika

\bigskip

\noindent \textit{Druh práce:} Bakalářská práce

\bigskip

\noindent \textit{Vedoucí práce:} Ing. Tomáš Pevný, Ph.D., Cisco systems, Inc.

\bigskip

\noindent \textit{Konzultant:} Mgr. Petr Somol, Ph.D., Cisco systems, Inc.

\bigskip

\noindent \textit{Abstrakt:}
Současné přístupy k detekci nežádoucího software sledováním síťového provozu klientů využívají ručně navržených příznaků jako části modelu. Tento přístup má několik nevýhod. Tato práce navrhuje plně automatický klasifikátor rozpoznávající aktivity malware na úrovni síťových spojení. K tomuto byl využit přístup pomocí multi-instančního učení. Součástí této práce je teoretické zavedení multi-instančního učení pomocí dvou rozdílných formalismů a souhrn dosavadních prací v oboru multi-instančního učení. Dále je popsána hierarchická struktura adresy URL jako vstupního objektu klasifikátoru. Je navržen model reflektující tuto inherentní strukturu a vysvětleno, jak využívá multi-instanční učení, v čem se liší a jak byl implementován pomocí umělých neuronových sítí. Jsou zde popsány metody sloužící k vyhodnocení kvality klasifikátoru a porovnání s předchozím dílem. Navržený klasifikátor je porovnán s nejlepším předchozím modelem a je srovnán vliv jednotlivých parametrů modelu na jeho kvalitu.

\bigskip

\noindent \textit{Klíčová slova:}
detekce malware, klasifikace síťového provozu, model adresy URL, multi-instanční učení, učení se reprezentace, strojové učení

\vfill

\begin{english}
	\begin{onehalfspace}
		\noindent \textit{Title:}

		\noindent \textbf{Hierarchical models of network traffic}
	\end{onehalfspace}

	\bigskip

	\noindent \textit{Author:} Marek Dědič

	\bigskip

	\todo{zkontrolovat}
	\noindent \textit{Abstract:}
	\textenglish{The current approach to the detection of unwanted software by monitoring client traffic uses hand-crafted features as part of the model. This approach has several disadvantages. This thesis proposes a fully automatic classifier which recognises malware activity on the level of network connections. The multi-instance learning approach was used in order to achieve this. As a part of this thesis the multi-instance learning was theoretically defined using two different formalisms and current work in this field was summarised. Subsequently, there was described the hierarchical structure of an URL address which was used as an input for the classifier. A model reflecting this inherent hierarchical structure was proposed and an explanation of how multi-instance learning was utilised and modified was presented together with the description of the implementation of the model using neural networks. Methods of classifier quality assessment were outlined. The presented classifier was compared with prior art and the influence of model parameters on its quality was assessed.}

	\bigskip

	\noindent \textit{Keywords:}
	\textenglish{machine learning, malware detection, multi-instance learning, network traffic classification, representation learning, URL address model}

\end{english}
