\chapter*{Závěr}
\addcontentsline{toc}{chapter}{Závěr}

V předložené práci byl navržen model pro automatickou detekci síťového provozu pocházejícího z aktivity malware, navrženy metody srovnání tohoto modelu s předchozími pracemi v oboru a provedeno experimentální vyhodnocení na reálných datech. Navržený model má srovnatelnou přesnost s nejlepším předchozím modelem, avšak výrazně vyšší odezvu. Tím byl dosažen a dokonce překonán vytyčený cíl práce, tj. aby navržený model měl alespoň stejnou kvalitu jako předchozí modely, využívající ručně vytvořeného seznamu příznaků. Navržený klasifikátor umožňuje detekci nežádoucího software s přesností i odezvou přesahující 90\%. Bylo tedy ukázáno, že přístup pomocí multi-instančního učení je v praxi využitelnou alternativou ke klasickému přístupu a poskytuje velmi dobré výsledky.

Jako možný další postup k vylepšení navrženého modelu se jeví opětovná aplikace mutli-instančního učení na tuto úlohu, výsledkem které by byl klasifikátor, který určuje přítomnost nežádoucího software na úrovni klientů (narozdíl od současného modelu, klasifikujícího na úrovni síťových spojení). Dalším možným krokem je opětovné použití multi-instančního učení na tento model, tentokrát na úrovni síťových toků, to jest sekvencí síťových spojení mířících ke stejné destinaci. Výsledný model by byl tvořen čtyřmi vnořenými multi-instančními úlohami. Jako další možné vylepšení navrženého modelu se jeví využití některých dalších polí HTTP hlavičky či techniky zrychlující proces učení. Mezi ně patří například využití přenosových funkcí obsahujících šum (srov. \cite{gulcehre_noisy_2016}), sebe-normalizujících neuronových sítí (srov. \cite{klambauer_self-normalizing_2017}) či využití učení pomocí syntetických gradientů (srov. \cite{jaderberg_decoupled_2016} a \cite{czarnecki_understanding_2017}). Mezi další možná rozšíření patří využití navženého přístupu při \BPenname{semi-supervised} či \BPenname{one-shot} učení.
