\documentclass[a4paper,11pt,oneside]{book}

\usepackage{polyglossia}
\usepackage{fontspec}
\usepackage{csquotes}
\usepackage{geometry}
\usepackage[intlimits]{amsmath}
\usepackage{amsfonts}
\usepackage{amsthm}
\usepackage{graphicx}
\usepackage{indentfirst}
\usepackage{url}
\usepackage[backend=biber,style=iso-authoryear,sortlocale=cs_CZ,autolang=other,bibencoding=UTF8]{biblatex}
\usepackage{setspace}
\usepackage{hyperref}
\usepackage[ocgcolorlinks]{ocgx2}
\usepackage{braket}
\usepackage{enumerate}
\usepackage{algorithm}
\usepackage{algorithmicx}
\usepackage{algpseudocode}
\usepackage{booktabs}
\usepackage{multirow}
\usepackage{float}

\geometry{tmargin=4cm,bmargin=3cm,lmargin=3cm,rmargin=2cm,headheight=0.8cm,headsep=1cm,footskip=0.5cm,marginparwidth=1.6cm}
\setdefaultlanguage{czech}
\setotherlanguage{english}
\setmainfont{TeX Gyre Termes}
\setcounter{secnumdepth}{3}
\addbibresource{zotero.bib}
\hypersetup{
	colorlinks,
	pdfencoding=auto,
	unicode=true,
	bookmarksopen=true,
	bookmarksopenlevel=3,
	citecolor=green,
	filecolor=blue,
	linkcolor=red,
	urlcolor=blue
}

% BibLaTeX fixes
\def\UrlBreaks{\do\/\do-}
\renewbibmacro{labeltitle}{}

\theoremstyle{plain}
\newtheorem{theorem}{Věta}[chapter]
\theoremstyle{definition}
\newtheorem{define}[theorem]{Definice}
\newtheorem{example}[theorem]{Příklad}
\theoremstyle{remark}
\newtheorem{remark}[theorem]{Poznámka}

\newcommand{\BPname}[1]{\textbf{#1}}
\newcommand{\BPenname}[1]{\textit{\textenglish{#1}}}

\newcommand{\BPspace}{\ensuremath{\mathcal}}
\newcommand{\BPfield}{\ensuremath{\mathbb}}
\newcommand{\BPset}{\ensuremath{\mathbb}}
\newcommand{\BPmat}{\ensuremath{\mathbf}}

\makeatletter
\renewcommand{\ALG@name}{Algoritmus}
\makeatother
\MakeRobust{\Call}
\algnewcommand\algorithmicparallelfor{\textbf{parallel for}}
\algdef{S}[FOR]{ParallelFor}[1]{\algorithmicparallelfor\ #1\ \algorithmicdo}
\algnewcommand\algorithmicbreak{\State\textbf{break}}
\algnewcommand\Break{\algorithmicbreak{}}

\newcommand{\result}[3]{%
	\begin{figure}[h]%
		\centering%
		\includegraphics[width=\textwidth]{images/results/#1}%
		\caption{#3}\label{#2}%
	\end{figure}%
}

\usepackage{todonotes}
