\chapter{Metody vyhodnocení}

Úloha popsaná v kapitole \ref{problem} je úlohou \BPname{binární klasifikace}. To znamená, že jejím cílem je zařadit vstupní objekty do jedné ze dvou tříd, označovaných jako pozitivní a negativní. Předpokládá se existence nějaké klasifikační funkce \( f : \BPspace X \to \left\{ -1, +1 \right\} \), která pro každý vstupní objekt určí příslušnou třídu, symbolizovanou hodnotou \( -1 \) nebo \( +1 \). Model popsaný v kapitole \ref{model} se snaží tuto neznámou klasifikační funkci aproximovat. Za účelem zhodnocení kvality této aproximace je v této kapitole nejprve zavedeno několik pojmů vztahujících se k binární klasifikaci a následně popsáno několik metod vizualizace kvality binárních klasifikátorů. \todo{Zkontrolovat po napsání}

\begin{define}
	Nechť \( r \in \left\{ -1, +1 \right\} \) je správným výsledkem pro nějaký vstupní objekt v úloze binární klasifikace. Nechť \( p \in \left\{ -1, +1 \right\} \) je aproximací \( r \). Odhad \( p \) nazýváme
	\begin{enumerate}[i.]
		\item \BPname{Pozitivní} (\BPenname{Prediction positive}) pokud \( p = +1 \).
		\item \BPname{Negativní} (\BPenname{Prediction negative}) pokud \( p = -1 \).
		\item \BPname{Pravdivě pozitivní} (\BPenname{True positive}) pokud \( r = +1 \land p = +1 \).
		\item \BPname{Pravdivě negativní} (\BPenname{True negative}) pokud \( r = -1 \land p = -1 \).
		\item \BPname{Falešně pozitivní} (\BPenname{False positive}) pokud \( r = -1 \land p = +1 \).
		\item \BPname{Falešně negativní} (\BPenname{False negative}) pokud \( r = +1 \land p = -1 \).
	\end{enumerate}
	Falešně pozitivní odhad je také nazýván \BPname{chybou prvního druhu}, falešně negativní odhad je také nazýván \BPname{chybou druhého druhu}.
\end{define}

Standardním způsobem, jak jsou tyto hodnoty zapisovány, je pomocí takzvané \BPname{kontingenční tabulky}. Podoba takovéto kontingenční tabulky je znázorněna v tabulce \ref{contingency_table}.

\begin{table}[h]
	\caption{Kontingenční tabulka}\label{contingency_table}
	\centering
	\begin{tabular}{llcc}
		\toprule
		&& \multicolumn{2}{c}{Reálný výsledek} \\
		\cmidrule(l){3-4}
		&& Pozitivní & Negativní \\
		\cmidrule{3-4}
		\multirow{2}{*}{Předpověď} & \multicolumn{1}{l|}{Pozitivní} & Počet pravdivě pozitivních odhadů & Počet falešně negativních odhadů \\
		& \multicolumn{1}{l|}{Negativní} & Počet falešně pozitivních odhadů & Počet pravdivě negativních odhadů \\
		\bottomrule
	\end{tabular}
\end{table}

\section{Indikátory kvality binárního klasifikátoru}\label{quality_indicators}
Cílem jakéhokoliv odhadu v úloze binární klasifikace je minimalizovat počet falešně pozitivních a falešně negativních odhadů. Ne vždy je ale možné minimalizovat obě tyto hodnoty a je třeba o nich uvažovat vždy najednou -- aproximace, která označí všechny objekty jako pozitivní nebude mít žádné falešně negativní odhady, ale v praxi nemá takováto aproximace žádnou hodnotu. Obdobně aproximace označující všechny objekty za negativní nebude produkovat žádné falešně pozitivní odhady, ale opět to není vhodná aproximace. -- Aby tedy bylo možno uvažovat o kvalitě nějakého binárního klasifikátoru, jsou dále zavedeny některé \BPname{indikátory kvality}.

\todo{Ja bych se tady rozepsal, cim jsou ty jednotlive miry inspirovane a proc byli zavedene.}

\begin{define}
	Nechť \( tp \) je počet pravdivě pozitivních odhadů nějakého binárního klasifikátoru na daném souboru dat. Nechť \( pp \) je celkový počet pozitivních odhadů tohoto binárního klasifikátoru (tedy počet pravdivě pozitivních a falešně pozitivních). Jako \BPname{přesnost} (\BPenname{precision}) tohoto klasifikátoru je označována hodnota
	\[ precision = \frac{tp}{pp} \]
\end{define}

\begin{define}
	Nechť \( tp \) je počet pravdivě pozitivních odhadů nějakého binárního klasifikátoru na daném souboru dat. Nechť \( rp \) je celkový počet pozitivních vzorků v datovém souboru (tedy počet pravdivě pozitivních a falešně negativních). Jako \BPname{odezva} (\BPenname{recall}) tohoto klasifikátoru je označována hodnota
	\[ recall = tpr = \frac{tp}{rp} \]
	Odezva je také nazývána \BPname{pravdivě pozitivní mírou} (\BPenname{true positive rate}).
\end{define}

\begin{define}
	Nechť \( fp \) je počet falešně pozitivních odhadů nějakého binárního klasifikátoru na daném souboru dat. Nechť \( rn \) je celkový počet negativních vzorků v datovém souboru (tedy počet pravdivě negativních a falešně pozitivních). Jako \BPname{falešně pozitivní míra} (\BPenname{false positive rate}) tohoto klasifikátoru je označována hodnota
	\[ fpr = \frac{fp}{rn} \]
\end{define}

\begin{define}
	Nechť \( fn \) je počet falešně negativních odhadů nějakého binárního klasifikátoru na daném souboru dat. Nechť \( rp \) je celkový počet pozitivních vzorků v datovém souboru (tedy počet pravdivě pozitivních a falešně negativních). Jako \BPname{falešně negativní míra} (\BPenname{false negative rate}) tohoto klasifikátoru je označována hodnota
	\[ fnr = \frac{fn}{rp} \]
\end{define}

\begin{define}
	Nechť \( precision \) je přesnost nějakékého binárního klasifikátoru a \( recall \) je jeho odezva. Jako \BPname{F-skóre} či také \BPname{F\textsubscript{1} skóre} tohoto klasifikátoru je označován harmonický průměr jeho přesnosti a odezvy, tedy
	\[ F_1 = 2 \, \frac{precision \cdot recall}{precision + recall} \]
\end{define}

\begin{theorem}\label{tpr+fnr}
	Platí
	\[ tpr + fnr = 1 \]
\end{theorem}

\todo{Název?}\section{Spojitá aproximace binárního klasifikátoru}\label{continuous_aprox}

Přestože v úloze binární klasifikace jsou objekty zařazované do jedné ze dvou tříd, může nastat případ, kdy aproximace klasifikační funkce má jiný obor hodnot než množinu \( \left\{ -1, +1 \right\} \). V obecném případě může být oborem hodnot této aproximace celý obor \( \BPfield R \). Je tedy třeba rozhodnout, které hodnoty binárního klasifikátoru budou považovány za pozitivní a které za negativní. Zřejmě se nabízí přístup, kdy je stanoven nějaký práh a všechny hodnoty vyšší než tento práh jsou považovány za pozitivní odhad, všechny hodnoty nižší než tento práh jsou považovány za negativní odhad. Indikátory kvality lze poté chápat jako funkce takovéhoto prahu. Volba správného prahu ovšem není triviálním úkolem.

Jedním z možných řešení problému volby prahu je přístup pomocí určování hodnoty indikátorů kvality binárního klasifikátoru pro všechny možné prahy. Vzhledem k předpokladu konečnosti klasifikovaného souboru dat jsou tyto indikátory (považované za funkce prahu) po částech konstantní a tedy postačuje je vyhodnotit v konečně mnoha bodech. Pro velké datasety (například dataset popsaný v kapitole \ref{dataset}) avšak ani toto není výpočetně možné. Zvoleným řešením je nalézt kvantily všech možných hodnot prahu a vyhodnotit indikátory kvality v těchto bodech.

\section{Křivky zobrazující vlastnosti binárního klasifikátoru}\label{evaluation_curves}
V následující podkapitole jsou popsány 4 křivky, pomocí kterých lze vizualizovat vlastnosti binárních klasifikátorů. V celé podkapitole jsou indikátory kvality považovány za funkce prahu.

\subsection{PR křivka}
\BPname{PR křivka} (\BPenname{Precision-Recall curve}) je křivkou zobrazující vztah mezi přesností a odezvou klasifikátoru pro různé prahy. PR křivka grafem množiny
\[ \left\{ \left( recall \left( x \right), precision \left( x \right) \right) \middle| x \in \BPfield R \right\} \]
Jde tedy o obor hodnot funkce \( \BPfield R \to \BPfield R^2 \), nikoliv o graf funkce \( \BPfield R \to \BPfield R \).

\subsection{ROC křivka}
\BPname{ROC křivka} (\BPenname{Receiver operating characteristic curve}) je křivkou zobrazující vztah mezi falešně pozitivní mírou a pravdivě pozitivní mírou klasifikátoru pro různé prahy. ROC křivka grafem množiny
\[ \left\{ \left( fpr \left( x \right), tpr \left( x \right) \right) \middle| x \in \BPfield R \right\} \]
V grafu ROC křivky je vyznačena i množina všech bodů \( \left( x, x \right) \), neboť náhodně volená klasifikační funkce by měla ROC křivku blízkou takovéto množině. Osa X grafu ROC křivky je v logaritmickém měřítku.

\subsection{DET křivka}
\BPname{DET křivka} (\BPenname{Detection error tradeoff curve}, srov. \cite{martin_det_1997}) je modifikací ROC křivky. DET křivka je grafem množiny
\[ \left\{ \left( fpr \left( x \right), fnr \left( x \right) \right) \middle| x \in \BPfield R \right\} \]
Z věty \ref{tpr+fnr} je zřejmé, že DET křivka je ROC křivkou otočenou podle osy Y \todo{Zdůraznit, že v bodě 0.5?}. Rozdílem ovšem je měřítko, v jakém je tato křivka vykreslována. Na obě osy grafu je aplikována funkce
\[ q \left( x \right) = \sqrt{2} \cdot erf^{-1} \left( 2x - 1 \right) \]
kde
\[ erf \left( x \right) = \frac{1}{\sqrt{\pi}} \int_{-x}^{x} e^{-t^2} \mathrm{d} t \]

\subsection{Křivka F-skóre}
Křivka F-skóre je grafem funkce \( x \mapsto F_1 \left( x \right) \).

\section{Proč vlastní implementace?}
K výpočtu křivek popsaných v kapitole \ref{evaluation_curves} bylo vyzkoušeno několik softwarových balíčků dostupných pro jazyk Julia (\cite{zea_roc.jl:_2015}, \cite{leeuwen_rocanalysis.jl:_2016} a \cite{lin_mlbase.jl:_2017}). Všechny tyto balíčky avšak neumožňovaly výpočet indikátorů kvality z důvodů příliš vysoké výpočetní náročnosti či paměťové náročnosti. Všechny tyto balíčky rovněž neumožňují paralelní výpočet indikátorů kvality. Proto byl navržen vlastní balíček s názvem EduNetsEvaluate.jl, který umožňuje výpočet indikátorů kvality popsaných v kapitole \ref{quality_indicators} i pro soubory dat tak velké jako ten, popsaný v kapitole \ref{dataset}. Výpočet všech indikátorů kvality je z velké části paralelizovaný, což umožňuje plně využít výpočetní kapacity moderních vícejádrových procesorů a procesorů využívajících technologii Hyper-threading.

\section{Implementace}
Následující podkapitola obsahuje přehled metod použitých k implementaci paralelního výpočtu PR křivky a ROC křivky.

Vzhledem k tomu, že použitý dataset je rozdělen do mnoha souborů, je možné každý takovýto soubor zpracovat zvlášť a následně agregovat výsledky. O velké části souborů je známo, že obsahují pouze objekty negativní třídy. Pokud je předpokládáno, že všechny objekty jsou seřazené podle hodnoty jejich předpovědi, při postupném procházení se při přechodu od negativního objektu k následujícímu negativnímu objektu nezmění hodnota odezvy. Tedy pokud bude PR křivka počítána pouze z pozitivních objektů, na její tvar to nebude mít žádný vliv. Pro ROC křivku platí stejné pozorování. Tím je umožněn výpočet vhodných hodnot prahů (ve smyslu kapitoly \ref{continuous_aprox}) pouze ze souborů, v nichž jsou nějaké pozitivní objekty. Tento výpočet je popsán v algoritmu \ref{thresholds}.

\begin{algorithm}
	\caption{Výpočet prahů}
	\label{thresholds}
	\begin{algorithmic}
		\Require $ mixed\_files $ \Comment Soubory obsahující pozitivní i negativní objekty
		\Statex
		\State $ thresholds \gets \Call{array}{} \left[ \, \right] \left[ \Call{size}{mixed\_files} \right] $ \Comment Pole polí
		\ParallelFor{$ file \in mixed\_files $}
			\State $ thresholds \left[ \, \right] \gets \Call{get\_predicted}{file} $
		\EndFor
		\State $ thresholds \gets \Call{concatenate}{thresholds} $
		\State $ thresholds \gets \Call{quantiles}{thresholds} $
	\end{algorithmic}
\end{algorithm}

Následně lze spočítat odezvu binárního klasifikátoru, k jejímuž výpočtu opět postačují pouze soubory, obsahující nějaké pozitivní objekty. Paralelizace tohoto výpočtu je umožněná	přístupem, kdy pro každý soubor je spočítán celkový počet pozitivních objektů v souboru a pro každý práh počet pravdivě pozitivních odhadů. Tyto hodnoty jsou poté sečteny přes všechny soubory a na jejich základě je spočítána odezva. Tento výpočet je popsán v algoritmu \ref{recall}. Pro výpočet přesnosti je již potřeba zohlednit všechny soubory včetně těch, které obsahují pouze negativní objekty. Pro každý soubor je zvlášť spočítán počet pravdivě pozitivních odhadů a celkový počet pozitivních odhadů. Tyto hodnoty jsou poté sečteny přes všechny soubory a na jejich základě je spočítána přesnost. Tento výpočet je analogický s výpočtem odezvy. 

\begin{algorithm}
	\caption{Výpočet odezvy}
	\label{recall}
	\begin{algorithmic}
		\Require $ mixed\_files $ \Comment Soubory obsahující pozitivní i negativní objekty
		\Require $ thresholds $
		\Statex
		\State $ RP \gets \Call{array}{} \left[ \Call{size}{mixed\_files} \right] $ \Comment Pole čísel
		\State $ TP \gets \Call{array}{} \left[ \Call{size}{thresholds} \right] \left[ \Call{size}{mixed\_files} \right] $ \Comment Pole polí
		\ParallelFor{$ file \in mixed\_files $}
			\State $ RP \left[ \, \right] \gets \Call{get\_real\_positives}{file} $
			\State $ TP \left[ \, \right] \gets \Call{get\_true\_positives}{file, thresholds} $
		\EndFor
		\State $ RP \gets \sum RP $
		\State $ TP \gets element-wise \sum TP $
		\State $ recall = \frac{TP}{RP} $
	\end{algorithmic}
\end{algorithm}

Pokud jsou objekty nejprve seřazeny podle hodnoty jejich předpovědi, lze poté počet pravdivě pozitivních i počet pozitivních odhadů počítat s asymptotickou složitostí \( \mathcal O \left( n \right) \), což spolu se složitostí řazení dává celkovou asymptotickou složitost \( \mathcal O \left( n \log n \right) \). Tento způsob efektivního výpočtu počtu pravdivě pozitivních odhadů je popsán v algoritmu \ref{TP}.

\begin{algorithm}
	\caption{Výpočet počtu pravdivě pozitivních odhadů se složitostí \( \mathcal O \left( n \right) \)}
	\label{TP}
	\begin{algorithmic}
		\Require $ predicted $ \Comment Odhady pro nějaký soubor objektů, seřazené vzestupně
		\Require $ real $ \Comment Skutečné třídy pro tento soubor ve stejném pořadí
		\Require $ thresholds $
		\Statex
		\State $ TP \gets \Call{array}{} \left[ \Call{size}{thresholds} \right] $
		\State $ TPcounter \gets \Call{count\_positives}{real} $
		\State $ THcounter \gets 1 $
		\For{$ i \gets 1 \; \text{to} \; \Call{size}{thresholds} $} \Comment Odhady menší než nejmenší z prahů
			\If{$ predicted \left[ 1 \right] < thresholds \left[ THcounter \right] $}
				\Break
			\EndIf
			\State $ TP \left[ THcounter \right] \gets TPcounter $
				\State $ THcounter \gets THcounter + 1 $
		\EndFor
		\If{$ real \left[ 1 \right] = +1 $}
			\State $ TPcounter \gets TPcounter - 1 $
		\EndIf
		\State $ j \gets 1 $
		\While{$ j < \Call{size}{predicted} - 1 $} \Comment Odhady mezi nejmenším a největším prahem
			\If{$ predicted \left[ j \right] < thresholds \left[ THcounter \right] \land predicted \left[ j + 1 \right] \geq thresholds \left[ THcounter \right] $}
				\State $ TP \left[ THcounter \right] \gets TPcounter $
				\State $ THcounter \gets THcounter + 1 $
			\Else
				\If{$ real \left[ j + 1 \right] = +1 $}
					\State $ TPcounter \gets TPcounter - 1 $
				\EndIf
				\State $ j \gets j + 1 $
			\EndIf
		\EndWhile
		\For{$ i \gets THcounter \; \text{to} \; \Call{size}{TP} $} \Comment Odhady větší než největší z prahů
			\State $ TP \left[ i \right] = 0 $
		\EndFor
	\end{algorithmic}
\end{algorithm}

Výpočet celkového počtu pozitivních odhadů je obdobný. Ve skutečné implementaci jsou tyto hodnoty počítány kvůli vyšší efektivitě najednou. Ze stejného důvodu je najednou počítána i přesnost s odezvou. Výpočet ROC křivky je obdobný jako výpočet PR křivky.

K výpočtu křivky F-skóre jsou využity hodnoty vypočítané pro PR křivku. K výpočtu DET křivky jsou využity hodnoty vypočítané pro ROC křivku a věta \ref{tpr+fnr}.
